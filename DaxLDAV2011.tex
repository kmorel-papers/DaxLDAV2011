% -*- latex -*-

%%%%%%%%%%%%%%%%%%%%%%%%%%%%%%%%%%%%%%%%%%%%%%%%%%%%%%%%%%%%%%%%%%%%%%%%%%%%%
% This beginning part of the preamble is specific to the vgtc document class.

\documentclass{vgtc}                          % final (conference style)
%\documentclass[review]{vgtc}                 % review
%\documentclass[widereview]{vgtc}             % wide-spaced review
%\documentclass[preprint]{vgtc}               % preprint
%\documentclass[electronic]{vgtc}             % electronic version

%% Uncomment one of the lines above depending on where your paper is
%% in the conference process. ``review'' and ``widereview'' are for review
%% submission, ``preprint'' is for pre-publication, and the final version
%% doesn't use a specific qualifier. Further, ``electronic'' includes
%% hyperreferences for more convenient online viewing.

%% Please use one of the ``review'' options in combination with the
%% assigned online id (see below) ONLY if your paper uses a double blind
%% review process. Some conferences, like IEEE Vis and InfoVis, have NOT
%% in the past.

%% Figures should be in CMYK or Grey scale format, otherwise, colour 
%% shifting may occur during the printing process.

%% These three lines bring in essential packages: ``mathptmx'' for Type 1 
%% typefaces, ``graphicx'' for inclusion of EPS figures. and ``times''
%% for proper handling of the times font family.

\usepackage{mathptmx}
\usepackage{graphicx}
\usepackage{times}

%% We encourage the use of mathptmx for consistent usage of times font
%% throughout the proceedings. However, if you encounter conflicts
%% with other math-related packages, you may want to disable it.

%% If you are submitting a paper to a conference for review with a double
%% blind reviewing process, please replace the value ``0'' below with your
%% OnlineID. Otherwise, you may safely leave it at ``0''.
\onlineid{0}

%% declare the category of your paper, only shown in review mode
\vgtccategory{Research}

%% allow for this line if you want the electronic option to work properly
\vgtcinsertpkg

%% In preprint mode you may define your own headline.
%\preprinttext{To appear in an IEEE VGTC sponsored conference.}

%% Paper title.

\title{A Proposed Framework for Data Analysis and Visualization at Extreme Scale}

%% This is how authors are specified in the conference style

%% Author and Affiliation (single author).
%%\author{Roy G. Biv\thanks{e-mail: roy.g.biv@aol.com}}
%%\affiliation{\scriptsize Allied Widgets Research}

%% Author and Affiliation (multiple authors with single affiliations).
%%\author{Roy G. Biv\thanks{e-mail: roy.g.biv@aol.com} %
%%\and Ed Grimley\thanks{e-mail:ed.grimley@aol.com} %
%%\and Martha Stewart\thanks{e-mail:martha.stewart@marthastewart.com}}
%%\affiliation{\scriptsize Martha Stewart Enterprises \\ Microsoft Research}

%% Author and Affiliation (multiple authors with multiple affiliations)
\author{Roy G. Biv\thanks{e-mail: roy.g.biv@aol.com}\\ %
        \scriptsize Starbucks Research %
\and Ed Grimley\thanks{e-mail:ed.grimley@aol.com}\\ %
     \scriptsize Grimley Widgets, Inc. %
\and Martha Stewart\thanks{e-mail:martha.stewart@marthastewart.com}\\ %
     \parbox{1.4in}{\scriptsize \centering Martha Stewart Enterprises \\ Microsoft Research}}

%% A teaser figure can be included as follows, but is not recommended since
%% the space is now taken up by a full width abstract.
%\teaser{
%  \includegraphics[width=1.5in]{sample.eps}
%  \caption{Lookit! Lookit!}
%}

%% Abstract section
\abstract{ Experts agree that the exascale machine will comprise processors
  that contain many cores, which in turn will necessitate a much higher
  degree of parallelism.  Software will require a minimum of a 1000 times
  more concurrency.  Most parallel analysis and visualization algorithms
  today work by partitioning data and running mostly serial algorithms
  concurrently on each data partition.  Although this approach lends itself
  well to the parallelism of current high performance computing, it does
  not exhibit the appropriate pervasive parallelism required for exascale
  computing. The data partitions are too small and the overhead of the
  threads too large to make effective use of all the cores in an extreme
  scale machine.  This paper introduces a new visualization framework
  designed to exhibit the pervasive parallelism necessary for extreme scale
  machines.  We demonstrate the use of this system on a GPU processor,
  which we feel is the best analog to an exascale node that we have
  available today.  }

%% ACM Computing Classification System (CCS). 
%% See <http://www.acm.org/class/1998/> for details.
%% The ``\CCScat'' command takes four arguments.

\CCScatlist{
  \CCScat{D.1.3}{Software}{Programming Techniques}{Concurrent Programming}
}

%% Copyright space is enabled by default as required by guidelines.
%% It is disabled by the 'review' option or via the following command:
% \nocopyrightspace

% End of vgtc-specific portion of the preamble.
%%%%%%%%%%%%%%%%%%%%%%%%%%%%%%%%%%%%%%%%%%%%%%%%%%%%%%%%%%%%%%%%%%%%%%%%%%%%%

\usepackage{amsfonts}
\usepackage{amssymb}
\usepackage{amsmath}
\usepackage{graphicx}
\usepackage{varioref}
\usepackage{fancyvrb}
\usepackage{ifthen}
\usepackage{cite}
\usepackage{subfigure}
\usepackage{xspace}
\usepackage{hyperref}

\usepackage{color}
\definecolor{yellow}{rgb}{1,1,0}
\definecolor{black}{rgb}{0,0,0}
\definecolor{ltcyan}{rgb}{.75,1,1}
\definecolor{red}{rgb}{1,0,0}

% Cite commands I use to abstract away the different ways to reference an
% entry in the bibliography (superscripts, numbers, dates, or author
% abbreviations).  \scite is a short cite that is used immediately after
% when the authors are mentioned.  \lcite is a full citation that is used
% anywhere.  Both should be used right next to the text being cited without
% any spacing.
\newcommand*{\lcite}[1]{~\cite{#1}}
\newcommand*{\scite}[1]{~\cite{#1}}

\newcommand*{\keyterm}[1]{\emph{#1}}

\newcommand{\sticky}[1]{{\color{red}\textsc{[#1]}}}

\begin{document}

%% VGTC-specific:
%% The ``\maketitle'' command must be the first command after the
%% ``\begin{document}'' command. It prepares and prints the title block.

%% the only exception to this rule is the \firstsection command
\firstsection{Introduction}

\maketitle

%% \section{Introduction} 
\label{sec:Introduction}

Most of today's visualization libraries and applications are based off of
what is known as the \keyterm{visualization
  pipeline}\lcite{Haeberli88,Lucas92}.  The visualization pipeline is the
key metaphor in many visualization development systems such as the
Visualization Toolkit (VTK)\lcite{VTKBook}, SCIRun\lcite{SCIRunPaper}, the
Application Visualization System (AVS)\lcite{AVSPaper},
OpenDX\lcite{OpenDXPaper}, and Iris Explorer\lcite{IRISExplorerPaper}.  It
is also the internal mechanism or external interface for may end-user
visualization applications such as ParaView\lcite{ParaViewGuideBook},
VisIt\lcite{VisItBook}, VisTrails\lcite{VisTrailsPaper},
MayaVi\lcite{MayaViPaper}, VolView\lcite{VolViewBook},
OsiriX\lcite{OsiriXPaper}, 3D Slicer\lcite{3DSlicerPaper}, and
BioImageXD\lcite{BioImageXDPaper}.

In the visualization pipeline model, algorithms are encapsulated as
\keyterm{filter} components with inputs and outputs.  These filters can
be combined by connecting the outputs of one filter to the inputs of
another filter.  The visualization pipeline model is popular because it
provides a convenient abstraction that allows users to combine algorithms
in powerful ways.

Although the visualization pipeline lends itself well to the concurrency of
current high performance
computing\lcite{Moreland08,Patchett09,Pugmire08,White05}, its structure
prohibits the necessary extreme parallelism required for exascale
computers.  This paper describes the design of the \keyterm{Dax toolkit} to
perform Data Analysis at Extreme scales.  The computational unit of this
framework is a \keyterm{functor}\sticky{Need to decide on nomenclature and
  do search/replace.}, a single operation on a small piece of data.
Functors can be combined in much the same way as filters, but their light
weight, lack of state, and small data access make them more suitable for
the massive concurrency required by exascale computers and associated
multi- and many-core processors.

\section{Errata}

To understand why, consider the sobering comparison between the
Jaguar XT5 partition, a current petascale machine, and the projections for
an exascale machine given by the International Exascale Software Project
RoadMap\lcite{ExascaleRoadMap} given in Table~\ref{table:PetaExaCompare}.
Because processor clock rates are not increasing, an exascale computer
requires a thousand-fold increase in the number of cores.  Furthermore,
trends in processor design suggest that these cores must be hyperthreaded
in order to keep them executing at full efficiency.  In all, to drive a
complete exascale machine will require between one and 10 billion
concurrently running threads.

\begin{table}
  \centering
  \caption{Comparison of characteristics between petascale and projected
    exascale machines.}
  \label{table:PetaExaCompare}
  \begin{tabular}{llll}
    & Jaguar -- XT5 & Exascale & Increase \\
    \hline
    Cores & 224,256 & \parbox{.7in}{100 million\\ \hspace*{6pt} -- 1 billion} & $\sim{}1,000\times$ \\
    Threads & 224,256 way & 1 -- 10 billion way & $\sim{}50,000\times$ \\
    Memory & 300 Terabytes & 128 Petabytes & $\sim{}500\times$
  \end{tabular}
\end{table}

Consider the problems of running our current visualization tools.  Most of
our current tools rely on MPI for concurrency.  An MPI process has the
overhead of a running program with its own memory space.  A common process
has an overhead of about twenty megabytes.  Running on the entirety of
Jaguar yields an overhead of about 4 terabytes, less than two percent of
the overall available memory.  In contrast, the overhead for using MPI
processes for all the concurrency on an exascale machine requires up to 200
petabytes, possibly exceeding the total memory on the system in overhead
alone.

Even getting around problems with the overhead of MPI, the visualization
pipeline still has scheduling problems at this level of concurrency.
Consider using 

HEREHEREHERE

The focus of this project is to design and implement a new
framework for visualization.  The computational unit of this framework is a
\keyterm{functor}, a single operation on a small piece of data.  Functors
can be combined in much the same way as filters, but their light weight,
lack of state, and small data access make them more suitable for the
massive concurrency required by exascale computers.

%% VGTC-specific section command.
\acknowledgements{ASCR Attribution.

  Part of this work was performed by Sandia National Laboratory.
  Sandia National Laboratories is a multi-program laboratory operated by
  Sandia Corporation, a wholly owned subsidiary of Lockheed Martin
  Corporation, for the U.S. Department of Energy's National Nuclear
  Security Administration.}

\bibliographystyle{abbrv}
\bibliography{DaxLDAV2011}

\end{document}
